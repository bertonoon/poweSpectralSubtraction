\documentclass[11pt,a4paper,twoside]{article}

\usepackage{amsmath} 
\usepackage{polski}
\usepackage[utf8]{inputenc}
\usepackage{latexsym}
\usepackage{multicol} 
\usepackage[nottoc]{tocbibind}
%\usepackage[unicode=true]{hyperref}
\usepackage{graphicx}
\usepackage{float}
\usepackage{siunitx}
\usepackage{hyperref}
\usepackage{tabto}
\usepackage{caption}
%\usepackage{showframe}
\usepackage{fullpage}
\usepackage{listingsutf8}
\usepackage[margin=2.54cm]{geometry}
\usepackage[singlelinecheck=false % <-- important
]{caption}

\lstset{ 
	commentstyle=\color{mygreen},    % comment style
	extendedchars=true,              % lets you use non-ASCII characters; for 8-bits encodings only, does not work with UTF-8
	firstnumber=1,                % start line enumeration with line 1000
	frame=single,	                   % adds a frame around the code
	language=Matlab,                 % the language of the code
	numbers=left,                    % where to put the line-numbers; possible values are (none, left, right)
	numbersep=5pt,                   % how far the line-numbers are from the code
	stepnumber=1,                    % the step between two line-numbers. If it's 1, each line will be numbered
	tabsize=2,	                   % sets default tabsize to 2 spaces
}

%opening
\title{Projekt II \\ \large Procesy Losowe - Teoria dla Praktyka}
\author{Klaudia Klębowska 160820 \\ Bartosz Filipów 160488}
\date{09.05.2019}

\begin{document}
	
\maketitle

\section{Power spectral subtraction}
Metoda odejmowania widmowego (ang. Spectral Subtraction) służy do oczyszczania zakłóceń oraz rekonstrukcji sygnałów mowy. 
Każdy nagrywany sygnał, oprócz użytecznej części posiada także pewien poziom szumu. Na wejściu programu mamy zaszumiony dyskretny sygnał $x[t]$. 

\subsection{DFT i IDFT}
Na początku zdefiniujmy dyskretną transformatę Fouriera (DFT) oraz odwrotną dyskretną transformatę Fouriera (IDFT), które będą potrzebne w dalszych krokach algorytmu.

\begin{equation}	
\mathrm{dft}(x[t])=X[\omega_{i}]= \sum_{t=1}^{N-1} x[t]e^{-j\omega_{i}t},
\end{equation}gdzie: \\
$\omega_{i}=\frac{2\pi i}{M}$, \\
$ i = 0,1,..., M-1$ \\
$ M $ - długość sygnału. \\

\begin{equation}
\mathrm{idft}(X[\omega])=x[t]= \frac{1}{M}\sum_{i=1}^{M-1} X[\omega_{i}]e^{j\omega_{i}t},
\end{equation}
gdzie: \\
$t = 0,1,...,M-1$, \\
$M$ - długość sygnału. \\


\subsection{Estymacja mocy widmowej szumu.}
Nasz sygnał należy podzielić na ramki o szerokości $N$. Powinna być ona tak dobrana, aby nie była dłuższa, niż 10-20 ms w przypadku sygnału mowy. W pierwszym kroku należy wykorzystać ramkę, która zawiera sam szum. Oznaczmy ją jako $z[t]$ i estymujmy jej widmową gęstość mocy: 
\begin{equation}
	\widehat{S_{Z}[\omega_{i}]} = \frac{1}{N}|Z[\omega_{i}]|^2,
\end{equation} 
gdzie: \\
$i = 0,1,...,\frac{N}{2}$, \\
$Z[\omega_{i}] = \mathrm{dft}(z[t])$.\\

Jeżeli dostępne jest wiele ramek zawierających tylko szum, to można zastosować uśrednianie:
\begin{equation}
	\widehat{S_{Z}[\omega_{i}]} = \frac{1}{U}\sum_{i=1}^{N}\sum_{j=1}^{U} \widehat{S_{Z_{j}}[\omega_{i}]},
\end{equation} gdzie: \\
U - ilość ramek zawierających tylko szum.


\subsection{Estymacja mocy widmowej sygnału.}
Drugim krokiem jest estymacja mocy widmowej sygnału. W tym celu bierzemy pierwszą ramkę sygnału wejściowego. Oznaczmy ją jako $y[t]$. 
\begin{equation}
	\widehat{S_{Y}[\omega_{i}]} = \frac{1}{N}|Y[\omega_{i}]|^2,
\end{equation} gdzie: \\
$i = 0,1,...,\frac{N}{2}$, \\
$Y[\omega_{i}] = \mathrm{dft}(y[t])$.

\subsection{Estymacja funkcji gęstości widmowej mocy sygnału bezszumowego.}
Trzecim krokiem algorytmu jest estymacja funkcji gęstości widmowej mocy sygnału bezszumowego. W tym celu wykorzystujemy wyniki z równań (4) i (5):
\begin{equation}
	\widehat{S_{X}[\omega_{i}]} = 
	\left\{ \begin{array}{ll}
		\widehat{S_{Y}[\omega_{i}]}-\widehat{S_{Z}[\omega_{i}]} & 
		\textrm{gdy $\widehat{S_{Y}[\omega_{i}]}-\widehat{S_{Z}[\omega_{i}]}\geq 0$}\\
		 0 & \textrm{w pozostałych przypadkach}\\
	\end{array}\right.,
\end{equation} gdzie: \\
$i = 0,1,...\frac{N}{2}$.

\subsection{Filtr odszumiający}
Czwartym krokiem jest zaprojektowanie filtra, który odszumi nasz sygnał wejściowy: 
\begin{equation}
		\widehat{A}[\omega_{i}] = \sqrt{\frac{\widehat{S_{X}}[\omega_{i}] }{\widehat{S_{Y}}[\omega_{i}]}},
\end{equation} gdzie: \\
$i = 0,1,...,\frac{N}{2}$.
Filtr ten jest symetryczny, więc drugą część próbek wystarczy odbić względem próbki $\frac{N}{2}$:
Czwartym krokiem jest zaprojektowanie filtra, który odszumi nasz sygnał wejściowy: 
\begin{equation}
	\widehat{A}[\omega_{i}] = 	\widehat{A}[\omega_{N-1-i}],
\end{equation} gdzie: \\
$i = \frac{N}{2}+1, \frac{N}{2}+2,...,N-1$.


\subsection{Odszumianie}
Piątym krokiem jest wykorzystanie filtra zaprojektowanego w poprzednim etapie: 
\begin{equation}
	\widehat{X_{czysty}}[\omega_{i}] = \widehat{A}[\omega_{i}]\widehat{Y}[\omega_{i}]
\end{equation}
Na koniec wystarczy przeprowadzić odwrotną dyskretną transformatę Fouriera:
\begin{equation}
	\widehat{x_{czysty}}[t] =\mathrm{idft}(\widehat{X_{czysty}}[\omega_{i}]) 
\end{equation}
W efekcie otrzymujemy pierwszą ramkę pozbawioną szumu. Teraz należy wczytać kolejną ramkę i wrócić do drugiego kroku. Czynności te powtarzamy do ostatniej ramki sygnału wejściowego. 

\subsection{Rozszerzenie metody o technikę overlap-add}
W przypadku, gdy na końcach lub początkach każdej z ramki słychać nieciągłość sygnału można zastosować technikę overlap-add. W tym celu należy pobierać kolejne ramki o długości $N$ z przesunięciem $\frac{N}{2}$ oraz wykorzystać współczynnik $w[t]$:
\begin{equation}
	w[t] = 1 - \frac{|t-N|}{N}
\end{equation}
\begin{equation}
	\widehat{x_{overlap_{i}}}[t]=\widehat{x_{czysty_i}}[t+\frac{N}{2}]w[t] + \widehat{x_{czysty_{i+1}}}[t]w[t],
\end{equation} gdzie: \\
$N$ - szerokość ramki,\\ 
$t = 1,2,...,\frac{N}{2}$, \\
$i = 1,2,...,R$, \\
$R$ - liczba ramek na które został podzielony sygnał wejściowy. \\

Dzięki temu otrzymujemy kolejne ramki pozbawione szumu oraz nieciągłości sygnału na początku lub końcu ramki.

\section{Listing programu}
Program został napisany w środowisku Matlab.
\begin{lstlisting}
clear; clc;
format compact
format long

% Parametry
koniec      = 2^18; 
dl_ramka    = koniec/256; 
skal        = 10; 
ram_szum    = 30; 

% Czytanie z pliku audio
[y,Fs]  = audioread('Randka w ciemno.wav');
info    = audioinfo('Randka w ciemno.wav');
y1      = y(1:1:koniec);

% Generowanie bialego szumu
szum = wgn(koniec,1,-40);

y2 = y1+szum;
y3 = zeros(length(y2)+dl_ramka*ram_szum,1);

for i=1:1:(koniec+dl_ramka*ram_szum)
	if i <= dl_ramka*ram_szum
		y3(i) = szum(i);
	else 
		y3(i) = y2(i-dl_ramka*ram_szum);
	end
end 

% Krok 1
ramki   = ram_szum;
S_z     = zeros(dl_ramka/2,1);
N       = dl_ramka;
szum_usr= zeros(dl_ramka,1);
szum2   = zeros(N,1);

for i=1:1:ramki
	for j=1:1:N
		szum2(j) = szum(j+(N*(i-1)));
	end
	Z = fft(szum2);
	for j=1:1:(N/2)
		S_z(j) = S_z(j) + (1/N)*(abs(Z(j))^2);
	end
end

S_z = S_z./ram_szum;
	
% Krok 2
ramki   = (length(y3)/dl_ramka);
y4      = y3(1:1:length(y3));
S_y     = zeros(dl_ramka/2,1);
N       = dl_ramka;
S_x     = zeros(dl_ramka/2,1);
A       = zeros(N,1);
X       = zeros(N,1);
sygnal  = zeros(N,1);
wyjsciowy = zeros(length(y4),1);

	
for i=1:0.5:ramki
	for j=1:1:N
		sygnal(j) = y4(j+dl_ramka*(i-1));
	end
	Y = fft(sygnal);
	
	for j=1:1:(N/2)
		S_y(j) = (1/N)*(abs(Y(j))^2);
	end 
	
	% Krok 3 
	for j=1:1:N/2
		if (S_y(j) - skal*S_z(j) >= 0)
			S_x(j) = S_y(j) - skal*S_z(j);
		else
			S_x(j) = 0;
		end
	end
	
	% Krok 4
	for j=1:1:N
		if j <= N/2
			A(j) = sqrt(S_x(j)/S_y(j));
		else
			A(j) = A(N-j+1);
		end
	end

	for j=1:1:length(Y)
		X(j) = (A(j))*(Y(j));
	end

	Xodw = real(ifft(X));

	for j=1:1:dl_ramka
		k = j+(N*(i-1));
		wp = 1-(abs(j-N*0.5)/(N*0.5));
		wyjsciowy(k) = wyjsciowy(k)+wp*Xodw(j);
	end
end
sound(wyjsciowy,Fs)
\end{lstlisting}
\end{document}